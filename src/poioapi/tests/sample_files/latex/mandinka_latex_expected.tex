\documentclass[a4paper,11pt]{article}
\usepackage{ucs}
\usepackage[utf8x]{inputenc}
\usepackage[T1]{fontenc}
\usepackage{gb4n}

\usepackage{latexsym}

\begin{document}
\ea
\glll
Musukéebâa níŋ a la maañóo le táata lóoñínóo la\\
Musu-kéebâa níŋ a la maañóo le táa-ta lóo-ñín-óo la\\
femme-âgé.D avec 3SG GEN jeune\_épouse.D FOC aller-ACPP bois-chercher-D OBL\\
\glt{} Une vieille femme et sa jeune co-épouse étaient allées chercher du bois.\\
\z
\ea
\glll
Kabíríŋ i ye i la lôo lándi sa□a mínínta maañóo la lóosítóo la bari wo máŋ a lóŋ\\
Kabíríŋ i ye i la lôo lá-ndi sa□a mínín-ta maañóo la lóo-sít-óo la bari wo máŋ a lóŋ\\
quand 3PL ACPP 3PL GEN bois.D être\_posé-CAUS serpent.D s’enrouler-ACPP jeune\_épouse.D GEN bois-attacher-D OBL mais DEM ACPN 3SG savoir\\
\glt{} Quand elles ont posé leur bois [avant de le charger], un serpent s’est enroulé autour du fagot de la jeune épouse, mais celle-ci ne s’en est pas aperçue.\\
\z
\ea
\glll
Musukeebaamâa ñáa be sǎa kaŋ míŋ be míníndiŋ a la lóosítôo bála a ye a fo a ye ko Níŋ yunduyóndóo sonta ŋ si ŋ kuu janníŋ ŋ be tábírôo kéla\\
Musu-keebaa-mâa ñáa be sǎa kaŋ míŋ be mínín-diŋ a la lóo-sít-ôo bála a ye a fo a ye ko Níŋ yunduyónd-óo son-ta ŋ si ŋ kuu janníŋ ŋ be tábí-r-ôo ké-la\\
femme-âgé-SELECT.D œil.D COPLOC serpent.D sur REL COPLOC s’enrouler-RES 3SG GEN bois-attacher-D CONT 3SG ACPP 3SG dire 3SG BEN QUOT si youndouyondo-D être\_d’accord-ACPP 1PL POT REFL laver avant\_que 1PL COPLOC cuire-ANTIP-D faire-INF\\
\glt{} La vieille avait son regard fixé sur le serpent qui était enroulé au fagot de la jeune co-épouse, elle lui a répondu, «Si le youndouyondo est d’accord, nous pourrons nous laver avant de faire à manger.»\\
\z
\end{document}